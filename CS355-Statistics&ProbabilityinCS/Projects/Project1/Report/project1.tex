\documentclass[12pt]{article}

\usepackage[margin=1in]{geometry}
\usepackage{graphicx}
\usepackage{amsmath}
\usepackage{adjustbox}

\begin{document}
	\begin{titlepage}
		\begin{center}
			\hspace{0pt}
				\vfill
					\Huge CS 355: Program 1\\
					\Large Josh Patton \& Vaishak Menon\\
					\Large 03/02/2023
				\vfill
			\hspace{0pt}
            \small We declare that we have completed this assignment in accordance with 
            the UAB Academic Integrity Code and the UAB CS Honor Code. We have 
            read the UAB Academic Integrity Code and understand that any breach 
            of the Code may result in severe penalties.
            We  also  declare  that  the  following  percentage  distribution 
            faithfully  represents  individual  group  members’  contributions  to 
            the completion of the assignment\\
            \small Vaishak Menon, 50\% contribution, Partial Code Work and Partial Report Work, VM, 03/02/2023
            \small Josh Patton, 50\% contribution, Partial Code Work and Partial Report Work, JP, 03/02/2023
		\end{center}
	\end{titlepage}
	\newpage
	\begin{center}
	\Large Report\\
	\end{center}
    \begin{enumerate}
        \item Functionality:
        \subitem The code has 3 main functions: shuffle, splitArray, and plotR. The shuffle function will create an array
        of size n based on the given bounds. It will then call splitArray to get the properly shuffled array and 
        then compute the correlation coefficient on every shuffle done. After the shuffling is complete, we call plotR to 
        plot the correlation for each run.
    \end{enumerate}
    \begin{center}
        \small First Run
    \end{center}
    \begin{enumerate}
        \item Plot r with respect to the times of shuffling. After how many shuffles are the cards 
        in the most random order? (That is, when is r at a minimum value?)
        \subitem The cards are in the most random order at shuffle 3 and 11.\\
        \begin{minipage}[t]{\linewidth}
            \centering
            \adjustbox{valign=t halign=t}
            {
              \includegraphics[width=.7\linewidth]{Figure_1.png}
            }
            \medskip       
        \end{minipage}
        \item Do the cards return to their original order?  After how many runs?
        \subitem They eventually return to the original order at around run 6, 8, and 11.
        \item Is a total of 15 runs enough to return to the original order?
        \subitem It is enough runs to return to 15 as shown in the graph above.
        \item If not, does it appear that the cards will return to their original order in a relatively 
        small number of shuffles?
        \subitem In this case, it does return and it continues to look like it will reach again.
    \end{enumerate}
    \newpage
    \begin{center}
        \small Second Run
    \end{center}
    \begin{enumerate}
        \item Plot r with respect to the times of shuffling. After how many shuffles are the cards 
        in the most random order? (That is, when is r at a minimum value?)
        \subitem The cards are in the most random order at shuffle 9, which is the lowest point for this graph.\\
        \begin{minipage}[t]{\linewidth}
            \centering
            \adjustbox{valign=t halign=t}
            {
              \includegraphics[width=.7\linewidth]{Figure_2.png}
            }
            \medskip       
        \end{minipage}
        \item Do the cards return to their original order?  After how many runs?
        \subitem The cards do not return to their original order, based on the correlations calculated.
        \item Is a total of 15 runs enough to return to the original order?
        \subitem It is not enough to get back to the original order within 15 runs.
        \item If not, does it appear that the cards will return to their original order in a relatively 
        small number of shuffles?
        \subitem In this case, it does not appear to be rebounding at a consistent pace to return to the original order.
    \end{enumerate}
    \newpage
    \begin{center}
        \small Third Run
    \end{center}
    \begin{enumerate}
        \item Plot r with respect to the times of shuffling. After how many shuffles are the cards 
        in the most random order? (That is, when is r at a minimum value?)
        \subitem The cards are in the most random order at shuffle 9, which is the lowest point for this graph.\\
        \begin{minipage}[t]{\linewidth}
            \centering
            \adjustbox{valign=t halign=t}
            {
              \includegraphics[width=.7\linewidth]{Figure_3.png}
            }
            \medskip       
        \end{minipage}
        \item Do the cards return to their original order?  After how many runs?
        \subitem The cards do not return to their original order, based on the correlations calculated.
        \item Is a total of 15 runs enough to return to the original order?
        \subitem It is not enough to get back to the original order within 15 runs.
        \item If not, does it appear that the cards will return to their original order in a relatively 
        small number of shuffles?
        \subitem In this case, it does not appear to be rebounding at a consistent pace to return to the original order.
    \end{enumerate}
    \newpage
    \begin{center}
        \small Fourth Run
    \end{center}
    \begin{enumerate}
        \item Plot r with respect to the times of shuffling. After how many shuffles are the cards 
        in the most random order? (That is, when is r at a minimum value?)
        \subitem The cards are in the most random order at shuffle 6, which is the lowest point for this graph.\\
        \begin{minipage}[t]{\linewidth}
            \centering
            \adjustbox{valign=t halign=t}
            {
              \includegraphics[width=.7\linewidth]{Figure_4.png}
            }
            \medskip       
        \end{minipage}
        \item Do the cards return to their original order?  After how many runs?
        \subitem The cards do not return to their original order, based on the correlations calculated.
        \item Is a total of 15 runs enough to return to the original order?
        \subitem It is does reach the original order at around shuffle 11 and 13.
        \item If not, does it appear that the cards will return to their original order in a relatively 
        small number of shuffles?
        \subitem In this case it does appear within the graph that the original order is reached based on correlation.
    \end{enumerate}
\end{document}